\documentclass[a4paper]{article}
\usepackage[english]{babel}
\usepackage[utf8]{inputenc}
\usepackage[left=4.0cm, right=4.0cm]{geometry}
\usepackage{amsmath}
\usepackage{graphicx}
\usepackage{fancyhdr}
\usepackage{hyperref}
\usepackage{enumerate}
\usepackage{amssymb}
\usepackage{amsthm}
\usepackage{tikz-cd}
\usepackage{ytableau}

\newcommand{\R}{\mathbb{R}}
\newcommand{\Z}{\mathbb{Z}}
\newcommand{\N}{\mathbb{N}}
\newcommand{\C}{\mathbb{C}}
\newcommand{\Q}{\mathbb{Q}}
\newcommand{\F}{\mathbb{F}}
\newcommand{\E}{\mathbb{E}}
\renewcommand{\P}{\mathbb{P}}
\newcommand{\f}[1]{\text{#1}}
\newcommand{\<}{\langle}
\renewcommand{\>}{\rangle}
\renewcommand{\^}{\wedge}
\renewcommand{\v}{\vee}


\pagestyle{fancy}
\fancyhf{}
\lhead{\today}
\chead{CS 70}
\rhead{William Wang}
\cfoot{\thepage}

\makeatletter
\renewcommand{\@seccntformat}[1]{}
\makeatother

% \setcounter{secnumdepth}{0}
\setlength{\parindent}{0pt}
\setlength{\parskip}{8pt}

\title{\textbf{CS 70}}
\author{\large Taught by \textit{Hulett}\\
\textit{Lecture 1 Notes} by William Wang}
\date{6/24/19}

\begin{document}

\maketitle
\newpage
\underline{Lecture}\\
Proposition: A statement that is unambiguously true or false.

\qquad Example: 2 + 2 = 4\\
\hline
$P \^ Q$: Conjunction, "both $P$ and $Q$"\\
$P \v Q$: Disjunction, "at least one of $P$ or $Q$"\\
$\neg P$: Negation, "not P"\\
\hline
De Morgan's Laws: Anything written with an '$\^$' can be written with a '$\v$'. Fully Expressive.\\
Corollary: '$\v$' and '$\neg$' is enough to be fully expressive.\\

\newpage
\underline{Discussion}\\
$A \implies B$ is true for any $A = False$ because it is vacuously true.\\



\end{document}