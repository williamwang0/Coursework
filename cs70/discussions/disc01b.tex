\documentclass[a4paper]{article}
\usepackage[english]{babel}
\usepackage[utf8]{inputenc}
\usepackage{enumitem}
\usepackage[left=4.0cm, right=4.0cm]{geometry}
\usepackage{amsmath}
\usepackage{graphicx}
\usepackage{fancyhdr}
\usepackage{hyperref}
\usepackage{enumerate}
\usepackage{amssymb}
\usepackage{amsthm}
\usepackage{tikz-cd}
\usepackage{ytableau}

\newcommand{\R}{\mathbb{R}}
\newcommand{\Z}{\mathbb{Z}}
\newcommand{\N}{\mathbb{N}}
\newcommand{\C}{\mathbb{C}}
\newcommand{\Q}{\mathbb{Q}}
\newcommand{\F}{\mathbb{F}}
\newcommand{\E}{\mathbb{E}}
\renewcommand{\P}{\mathbb{P}}
\newcommand{\f}[1]{\text{#1}}
\newcommand{\<}{\langle}
\renewcommand{\>}{\rangle}
\renewcommand{\^}{\wedge}
\renewcommand{\v}{\vee}


\pagestyle{fancy}
\fancyhf{}
\lhead{\today}
\chead{CS 70}
\rhead{William Wang}
\cfoot{\thepage}

\makeatletter
\renewcommand{\@seccntformat}[1]{}
\makeatother

% \setcounter{secnumdepth}{0}
\setlength{\parindent}{0pt}
\setlength{\parskip}{8pt}

\title{\textbf{CS 70 sp19}}
\author{\large Taught by Rao, Ayazifar\\
Discussion 00b by William Wang}
\date{June 3, 2019}

\begin{document}

\maketitle
\begin{enumerate}
    \item We proceed by contraposition. Assume that there exists $a, b, c, d$ such that $a \geq c$ and $b \geq d$. Clearly, $a + b \geq c + d$. \qed
    \item For n to be odd, m must also be odd. In other words, $m = 2x + 1$ for $x \in \N$. Thus, $n = m^2 = (2x + 1)^2 = 4x^2 + 4x + 1 = 4x(x + 1) + 1$. Clearly, either $x$ or $x + 1$ is even, so we can express $x(x + 1)$ as $2k$ for $k \in \Z$. Thus, $n = 8k + 1$, which is what we desired to prove. \qed
    \item We proceed by contradiction. Assume that each person at the party has a unique number of friends. Clearly it is impossible for anyone at the party to have greater than $n-1$ friendships. Thus, in order to each person to have a unique number of friends, the number of friendships must be described by the following set: $\{0, 1, 2, ..., n-2, n-1\}$. This might seem possible at first, however upon further inspection, we see that the individual with $n-1$ friends must be friends with \textit{everyone} else at the party. Since there is an individual at this party with 0 friends, and friendships are pairwise, we can conclude this is impossible. \qed
    \item We proceed by contradiction. Assume that $2^\frac{1}{n}$ is rational for an integer $n \geq 3$. Then, by definition, $2^\frac{1}{n} = \frac{x}{y}$ for $x, y \in \Z$. Raising both sides of the equation to the $n\textsuperscript{th}$ power, we have: $2y^n = x^n$. Expanding, we get $y^n + y^n = x^n$, which is impossible for $n \geq 3$ according to Fermat's Last Theorem. \qed
    \item We proceed by cases. By the definition of a prime number, $m$ cannot be divisible by 2 or 3. Assume there exists a prime number $l > 3$ such that $l$ is not in the form $6k + 1$ or $6k - 1$. Then, $l$ must be one of the following:
    \begin{enumerate}
        \item Case 1: $6k + 2 = 2(3k + 1)$
        \item Case 2: $6k + 3 = 3(2k + 1)$
        \item Case 3: $6k + 4 = 2(3k + 2)$
    \end{enumerate}
    Note that $6k + 5$ is equivalent to $6(k + 1) - 1$.\\
    In case 1 and case 3, $l$ is divisible by 2. In case 2, $l$ is divisible by 3. Therefore $m$ must be in the form $6k + 1$ or $6k - 1$. \qed
\end{enumerate}
\newpage


\end{document}