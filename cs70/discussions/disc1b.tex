\documentclass[a4paper]{article}
\usepackage[english]{babel}
\usepackage[utf8]{inputenc}
\usepackage[left=4.0cm, right=4.0cm]{geometry}
\usepackage{amsmath}
\usepackage{graphicx}
\usepackage{fancyhdr}
\usepackage{hyperref}
\usepackage{enumerate}
\usepackage{amssymb}
\usepackage{amsthm}
\usepackage{tikz-cd}
\usepackage{ytableau}
\usepackage{makecell}

\newcommand{\R}{\mathbb{R}}
\newcommand{\Z}{\mathbb{Z}}
\newcommand{\N}{\mathbb{N}}
\newcommand{\C}{\mathbb{C}}
\newcommand{\Q}{\mathbb{Q}}
\newcommand{\F}{\mathbb{F}}
\newcommand{\E}{\mathbb{E}}
\renewcommand{\P}{\mathbb{P}}
\newcommand{\f}[1]{\text{#1}}
\newcommand{\<}{\langle}
\renewcommand{\>}{\rangle}
\renewcommand{\^}{\wedge}
\renewcommand{\v}{\vee}


\pagestyle{fancy}
\fancyhf{}
\lhead{\today}
\chead{Class Name}
\rhead{William Wang}
\cfoot{\thepage}

\makeatletter
\renewcommand{\@seccntformat}[1]{}
\makeatother

% \setcounter{secnumdepth}{0}
\setlength{\parindent}{0pt}
\setlength{\parskip}{8pt}

\title{\textbf{CS 70 sp19}}
\author{\large Taught by Rao, Ayazifar\\
Discussion 00b by William Wang}
\date{June 13, 2019}

\begin{document}

\maketitle
\begin{enumerate}
    \item 
    \begin{tabular}{|c|c|c|}
    \hline
    Days & Women & Men \\
    \hline
    1 & \makecell{A \\ B \\ C} & \makecell{\textbf{1}, 3 \\ \textbf{2} \\ _} \\
    \hline
    2 & \makecell{A \\ B \\ C} & \makecell{\textbf{1} \\ \textbf{3}, 2 \\ _} \\
    \hline
    3 & \makecell{A \\ B \\ C} & \makecell{\textbf{2}, 1 \\ \textbf{3} \\ _} \\
    \hline
    4 & \makecell{A \\ B \\ C} & \makecell{\textbf{2} \\ \textbf{1}, 3 \\ _} \\
    \hline
    5 & \makecell{A \\ B \\ C} & \makecell{\textbf{2} \\ \textbf{1} \\ \textbf{3}} \\
    \hline
    \end{tabular}
    \\Algorithm completes in 5 days and the pairing on day 5 shown is the final result.
    \item
    \begin{enumerate}
        \item False. Assume there exists a woman and man who prefer each other. Assume this woman is preferred least among all other men. On day one, this woman is proposed to by the man who prefers her. This will be the only proposal this woman will receive.
        \item True. If a woman receives a proposal on day $k$, then for ever day $l$ for $l > k$, she receives a proposal from the man who proposed to her on day $k$ and potentially more from other men. If a proposal on day $k$ $\implies$ a proposal on day $l$, then no proposal on day $k$ $\implies$ no proposal on day $l$.
        \item 
    \end{enumerate}
\end{enumerate}
\newpage


\end{document}