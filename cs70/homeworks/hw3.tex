\documentclass[a4paper]{article}
\usepackage[english]{babel}
\usepackage[utf8]{inputenc}
\usepackage[left=4.0cm, right=4.0cm]{geometry}
\usepackage{amsmath}
\usepackage{graphicx}
\usepackage{fancyhdr}
\usepackage{hyperref}
\usepackage{enumerate}
\usepackage{amssymb}
\usepackage{amsthm}
\usepackage{tikz-cd}
\usepackage{ytableau}

\newcommand{\R}{\mathbb{R}}
\newcommand{\Z}{\mathbb{Z}}
\newcommand{\N}{\mathbb{N}}
\newcommand{\C}{\mathbb{C}}
\newcommand{\Q}{\mathbb{Q}}
\newcommand{\F}{\mathbb{F}}
\newcommand{\E}{\mathbb{E}}
\renewcommand{\P}{\mathbb{P}}
\newcommand{\f}[1]{\text{#1}}
\newcommand{\<}{\langle}
\renewcommand{\>}{\rangle}
\renewcommand{\^}{\wedge}
\renewcommand{\v}{\vee}


\pagestyle{fancy}
\fancyhf{}
\lhead{\today}
\chead{CS 70}
\rhead{William Wang}
\cfoot{\thepage}

\makeatletter
\renewcommand{\@seccntformat}[1]{}
\makeatother

% \setcounter{secnumdepth}{0}
\setlength{\parindent}{0pt}
\setlength{\parskip}{8pt}

\title{\textbf{CS 70}}
\author{\large Taught by Hulett, Yang\\
Homework 3 by William Wang}
\date{July 14, 2019}

\begin{document}

\maketitle
\newpage
Sundry: I worked on this homework by myself.\\
\begin{enumerate}
    \item
        \begin{enumerate}
            \item Using Fermat's Little Theorem, we have that $a^{p-1} \equiv 1$ (mod p). In other words, $a^{p-1} = kp + 1$ for some integer k. Then, $a^{p(p-1)} = (kp + 1)^p = p^2(j) + 1$ where j is some integer expression that can be obtained by expanding $(kp + 1)^p$. Clearly, this leaves a remainder of 1 in modulo $p^2$.
            \item By definition of d and e, we have that $ed = kp(p-1)q(q-1) + 1$ for some integer $k$. Then, $x^{ed} \equiv x^{kp(p-1)q(q-1)} * x$ (mod $p^2q^2$). By CRT and the property shown in part (a), we have that $x^{kp(p-1)q(q-1)} \equiv 1$ (mod $p^2q^2)$, so $x^{ed} \equiv 1 * x \equiv x$ (mod $p^2q^2$).
        \end{enumerate}
    \item
        It is trivial to factor N into $p$ and $q$ if $(p-1)(q-1)$ is known. Observe that $(p-1)(q-1) = pq - (p+q) + 1 = N - (p+q) + 1$\\ So $(p+q) = pq - (p-1)(q-1) + 1$ which consists of terms that we know. Consider the polynomial $f(x) = (x-p)(x-q) = x^2 - (p+q)x + pq$. If we can construct $f$, then Wolfram can find the roots and return $p$ and $q$. We have shown that if we know $(p-1)(q-1)$, then we know $p + q$ and $pq$ is public knowledge. Thus, knowing the value of $(p-1)(q-1)$ makes finding $p$ and $q$ trivial.
    \item
        \begin{enumerate}
            \item $p(x) = y_0$.
            \item $a_1 = \frac{y_1 - y_0}{x_1 - x_0}$
            \item $a_2 = \frac{y_2 - y_1 - y_0}{(x_2-x_0) * (x_2 - x_1)}$
            \item $a_i = \frac{y_i - \sum_{k=0}^{k-1} y_k}{\prod_{j=0}^{j-1} x_i - x_j}$
        \end{enumerate}
    \item
        \begin{enumerate}
            \item Use the first 5 bits from the message to interpolate a degree 4 polynomial $p(x)$. Use 10 points from $p$ and send those through Channel A. Use the last 3 bits from the message to do the same with a degree 2 polynomial and channel B.
            \item The value (2,3) is wrong because the point makes the polynomial degree $\geq 4$. The polynomial should only be degree 2 so you should reinterpolate the polynomial using all the other points.
            \item Bob can still figure out the message. Since Bob knows the polynomial should be degree 2, he can see that the point at $x = 2$ is wrong and use interpolation to recover the point at $x=2$.
            \item Alice has the following set of points: (0,-3), (1,-1), (2,1), (3,3)\\
            Bob has the following set of points: (0,-3), (1,-1), (2,$x$), (3,-3), (4,5)
        \end{enumerate}
    \item
        \begin{enumerate}
            \item $P(0) = 4, P(1) = 3, P(2) = 2$\\
            Using Lagrange interpolation,\\
            $\delta_0(x) = \frac{(x-1)(x-2)}{(0-1)(0-2)} = 3(x-1)(x-2)$ (mod 5)\\
            $\delta_1(x) = \frac{(x)(x-2)}{(1-0)(1-2)} = 4(x)(x-2)$ (mod 5)\\
            $\delta_2(x) = \frac{(x)(x-1)}{(2-0)(2-1)} = 3(x)(x-1)$ (mod 5)\\
            $P(x) = 4(3(x-1)(x-2)) + 3(4x(x-2)) + 2(3x(x-1))$ (mod 5)\\
            $P(x) = -x + 4$ (mod 5)\\
            $(c_0, c_1, c_2, c_3, c_4) = (4,3,2,1,0)$
            \item
            corrupt = (0,3,2,1,0)\\
            $E(x) = x$\\
            $Q(x) = ax^2 + bx + c$\\
            $i=0: c = 0$ (mod 5)\\
            $i=1: a+b+c = 3$ (mod 5)\\
            $i=2: 4a + 2b + c = 4$ (mod 5)\\
            $i=3: 9a + 3b + c = 3$ (mod 5)\\
            $i=4: 16a + 4b + c = 0$ (mod 5)
            \item
            Recover original message using $P = \frac{Q}{E}$\\
            $P(x) = \frac{-x^2 + 4x}{x} = 4-x$\\
            Plugging in $x = 0,1,2$ yields $P(x)$
        \end{enumerate}
    \item
        \begin{enumerate}
            \item We proceed with a direct proof\\
            If $f$ is countable then we can list $f_1,f_2,...$\\
            $F(1) = f_1(1) + 1$\\
            $\forall n \in \N, n \geq 2 \implies F(n) = F(n-1) + f_n(n) + 1$\\
            F is increasing. The set of increasing $f_n$ is uncountable.
            \item $f(n) = i$ is the set for all $i$. We have $f(1) \geq f(2) \geq f(3) ... $. The $f(i)$ eventually tend to a constant in the following manner: $\forall n \geq n_0 \implies f(n) = i$. $f$ is the a set that is decreasing so we have the mapping of $f \in F$
            \item  This is necessarily countable. If we choose a point in each disk that is a rational number, we get a set that can be represented with $\Q x \Q$. The disks are disjoint so all the numbers are unique. Therefore, we have a one-to-one association from the disk to the countable set and it is necessarily countable.
            \item This is uncountable. Evaluate the set of all circles centered at the point (0, 0). $\forall r \in \R$, we can construct a circle of radius r. None of these circles overlap so there is a one-to-one mapping of each circle to a value in $\R$ which is an uncountable set.
        \end{enumerate}
\end{enumerate}
\end{document}