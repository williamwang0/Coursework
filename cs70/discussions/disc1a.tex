\documentclass[a4paper]{article}
\usepackage[english]{babel}
\usepackage[utf8]{inputenc}
\usepackage{enumitem}
\usepackage[left=4.0cm, right=4.0cm]{geometry}
\usepackage{amsmath}
\usepackage{graphicx}
\usepackage{fancyhdr}
\usepackage{hyperref}
\usepackage{enumerate}
\usepackage{amssymb}
\usepackage{amsthm}
\usepackage{tikz-cd}
\usepackage{ytableau}

\newcommand{\R}{\mathbb{R}}
\newcommand{\Z}{\mathbb{Z}}
\newcommand{\N}{\mathbb{N}}
\newcommand{\C}{\mathbb{C}}
\newcommand{\Q}{\mathbb{Q}}
\newcommand{\F}{\mathbb{F}}
\newcommand{\E}{\mathbb{E}}
\renewcommand{\P}{\mathbb{P}}
\newcommand{\f}[1]{\text{#1}}
\newcommand{\<}{\langle}
\renewcommand{\>}{\rangle}
\renewcommand{\^}{\wedge}
\renewcommand{\v}{\vee}


\pagestyle{fancy}
\fancyhf{}
\lhead{\today}
\chead{CS 70}
\rhead{William Wang}
\cfoot{\thepage}

\makeatletter
\renewcommand{\@seccntformat}[1]{}
\makeatother

% \setcounter{secnumdepth}{0}
\setlength{\parindent}{0pt}
\setlength{\parskip}{8pt}

\title{\textbf{CS 70 sp19}}
\author{\large Taught by Rao, Ayazifar\\
Discussion 00a by William Wang}
\date{June 5, 2019}

\begin{document}

\maketitle
\begin{enumerate}
    \item 
    \begin{enumerate}
        \item We proceed with proof by induction. \\
        \textit{Base Case} (n = 3): $2^3 > 2(3) + 1$. The base case holds.\\
        \textit{Induction Hypothesis}: Assume that $2^k > 2k + 1$ for some $k > 2$.\\
        \textit{Inductive Step}: By the induction hypothesis, we have that \\$2\textsuperscript{k+1} > 2(k+1) + 1$. To prove our claim, it suffices to show that $2^k \geq 2$.\\
        By the principle of mathematical induction, the claim holds.
        \item We proceed with proof by induction.\\
        \textit{Base Case} (n = 1): $\frac{(1)(1+1)(2+1)}{6} = 1^2 = 1$. The base case holds.\\
        \textit{Induction Hypothesis}: Assume that $1^2 + 2^2 + ... + k^2 = \frac{(k)(k+1)(2k+1)}{6}$ for some $k \in \Z^+$.\\
        \textit{Inductive Step}: By the induction hypothesis, we have that $1^2 + 2^2 + ... + k^2 + (k+1)^2 = \frac{(k)(k+1)(2k+1)}{6} + (k+1)^2 = \frac{(k+1)(k+2)(2k+3)}{6}$.\\
        By the principle of mathematical induction, the claim holds.
        \item We proceed with proof by induction.\\
        \textit{Base Case} (n = 1): $(\frac{5}{4})(8) + 3^2 = 19$. The base case holds.\\
        \textit{Induction Hypothesis}: Assume that $(\frac{5}{4})(8^k) + 3\textsuperscript{3k - 1} \equiv 0\mod{19}$ for some $k \in \N^+$.\\
        \textit{Inductive Step}: By the induction hypothesis, we have that $(\frac{5}{4})(8\textsuperscript{k+1}) + 3\textsuperscript{3(k+1) - 1} = (\frac{5}{4})(8)(8^k) + (27)(3\textsuperscript{3k -1}) = 8(\textit{Base Case}) + 19(3\textsuperscript{3k-1}) \equiv 0\mod{19}$.\\ 
        By the principle of mathematical induction, the claim holds. 
    \end{enumerate}
    \item
    \begin{enumerate}
        \item 
    \end{enumerate}
\end{enumerate}
\newpage


\end{document}